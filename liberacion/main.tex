\documentclass{article}
\usepackage[spanish]{babel}
\usepackage[utf8]{inputenc}

\usepackage{mathtools}
\usepackage{amssymb}
\usepackage{amsmath}
\usepackage{amsfonts}
\usepackage{graphics}
\usepackage{listings}

\usepackage[margin=1cm]{geometry}

%\renewcommand*{\familydefault}{\sfdefault}

\begin{document}
\begin{center}
  \textsc{Examen de Liberación} \\
  \textsc{Auxiliatura de INF-111} \\
  \textsc{Todos los paralelos} \\~ \\~
  \textsc{Apellidos y Nombres:} \rule{6cm}{.1pt} \hfill
  \textsc{Docente, Auxiliar o paralelo: } \rule{3cm}{.1pt}

  \hrulefill
\end{center}
\textsc{Instrucciones:} Este examen tiene 3 secciones, cada sección vale 35\% del examen (sumando 105\%), cada sección tiene 2 problemas, elige el que más te guste. Luego de que se te entregue el examen, se explicarán las preguntas durante 10 minutos. Cuando la explicación acabe, tendrás 2 horas para resolverlo. ¡Mide bien tu tiempo y no te encapriches con una pregunta! \\
\textit{\textbf{Consejo:} si pasas más de 15 minutos en una pregunta, déjala e intenta la siguiente}.

\hrulefill

\textsc{Cadenas} \textit{Punteo: 35\% del examen}
\begin{enumerate}
\item \textsc{(Punteo: 100\% de la sección de Cadenas)}\\ Dada una cadena de texto $S$, y otra cadena de texto $Q$, haz un algoritmo que indique si $S$ y $Q$ son \textit{anagramas}. \\
  \textit{\textbf{NOTA:} Se dice que una cadena de texto $A$ es anagrama de otra cadena de texto $B$, si y solo si $B$ contiene todos los caracteres de $A$ (a excepción de los espacios).}\\
  \textit{\textbf{NOTA:} Puedes asumir que las cadenas de texto únicamente tendrán letras minúsculas sin acentos y espacios.}

  \begin{tabular} {p{8.5cm} p{8.5cm}}
    \begin{lstlisting} [breaklines=true]
      S="diabole in dracon limala asno"
      Q="leonardo da vinci la mona lisa"
      SON ANAGRAMAS
    \end{lstlisting} &    
    \begin{lstlisting} [breaklines=true]
      S="imagen"
      Q="enigma"
      SON ANAGRAMAS
    \end{lstlisting} \\
    \begin{lstlisting} [breaklines=true]
      S="imagen"
      Q="ennigma"
      NO SON ANAGRAMAS
    \end{lstlisting} &    
    \begin{lstlisting} [breaklines=true]
      S="asdf"
      Q="axdf"
      NO SON ANAGRAMAS
    \end{lstlisting}    
  \end{tabular}
  
\item \textsc{(Punteo: 100\% de la sección de Cadenas)}\\
  Dada una cadena de texto $S$, haz un algoritmo que haga un conteo de todos sus caracteres.\\
  \textit{\textbf{NOTA:} No olvides que todo caracter tiene un valor ASCII!! y puedes acceder a él ``casteándo'' el caracter a una variable de tipo entero, de la siguiente forma: (int) 'A' = 65. Puedes obtener un caracter a partir de su código ASCII, si ``casteas'' un número a una variable de tipo caracter, de la siguiente forma (char) 66 = 'B'}

  \begin{tabular} {p{8.5cm} p{8.5cm}}
    
    \begin{lstlisting} [breaklines=true]
      S="topcoder ftw!!!"
      't':2, 'o':2, 'p':1, 'c':1, 
      'd':1, 'e':1, 'r':1, ' ':1, 
      'f':1, 'w':1, '!':3
    \end{lstlisting} &
    
    \begin{lstlisting} [breaklines=true]
      S="abacaba"
      'a':4, 'b':2, 'c':1
    \end{lstlisting}
    
  \end{tabular}
  
\end{enumerate}

\hrulefill

\textsc{Vectores y Matrices} \textit{Punteo: 35\% del examen}
\begin{enumerate}
\item \textsc{(Punteo: 100\% de la sección de Vectores y Matrices)} \\
  Dado un vector $V$ ordenado, de tamaño $n+1$, con $n$ elementos, y un número $x$, insertar el elemento $x$ en $V$, de tal forma que $V$ permanezca ordenado.   
  
  \begin{tabular} {p{8.5cm} p{8.5cm}}
    
    \begin{lstlisting} [breaklines=true]
      V = {1, 2, 3, 4, 5, 6, 7, 8, 9, NULL}
      x = 5
      V = {1, 2, 3, 4, 5, 5, 6, 7, 8, 9}
    \end{lstlisting} &
    
    \begin{lstlisting} [breaklines=true]
      V = {-500, -320, 0, 180, 241}
      x = -319
      V = {-500, -320, -319, 0, 180, 241}
    \end{lstlisting}
    
  \end{tabular}
  
\newpage
\item \textsc{(Punteo: 100\% de la sección de Vectores y Matrices)} \\
  Haz un algoritmo que construya una matriz $M$ de tamaño $a \times b$ (no es una matriz cuadrada), cuyos elementos de la diagonal principal sean $1$s, y el resto sean $0$s. \\
  \textit{\textbf{NOTA:} Puedes asumir que $a$ siempre será múltiplo de $b$}

  \begin{tabular} {p{8.5cm} p{8.5cm}}
    
    \begin{lstlisting} [breaklines=true]
      a=10
      b=5
      M={{1, 1, 0, 0, 0, 0, 0, 0, 0, 0},
         {0, 0, 1, 1, 0, 0, 0, 0, 0, 0},
         {0, 0, 0, 0, 1, 1, 0, 0, 0, 0},
         {0, 0, 0, 0, 0, 0, 1, 1, 0, 0},
         {0, 0, 0, 0, 0, 0, 0, 0, 1, 1}}
    \end{lstlisting} &
    
    \begin{lstlisting} [breaklines=true]
      a=9
      b=3
      M={{3, 3, 3, 0, 0, 0, 0, 0, 0},
         {0, 0, 0, 3, 3, 3, 0, 0, 0},
         {0, 0, 0, 0, 0, 0, 3, 3, 3}}
    \end{lstlisting}
    
  \end{tabular}
  
\end{enumerate}

\hrulefill

\textsc{Lotes, Series y Sumatorias} \textit{Punteo: 35\% del examen}

\begin{enumerate}
\item \textit{(Punteo: 100\% de la sección Lotes, Series y Sumatorias)} \\
  hay un robot que solo puede caminar hacia adelante y hacia atrás, cada paso del robot mide exactamente $1cm$, y siempre que quiere cambiar de dirección, tiene que detenerse durante 1 segundo. Cada vez que da un paso, el robot gasta $r$ watts.

  El robot está caminando en un desierto perfecto en el que no hay obstáculos,  y todo el terreno es plano. El robot está en el punto 0 del desierto y tiene un dispositivo GPS que, cada vez que el robot se detiene durante más de medio segundo, te envía un informe de su posición (la posición es un número entero $x$, que mide la distancia desde el punto 0 hasta la posición en la que está el robot al momento que se detuvo.

  Haz un algoritmo que, dado un lote de $n$ números, y un número $r$ que indica los watts gastados por paso, te indique cuántos watts ha gastado el robot.
  
  \begin{tabular} {p{8.5cm} p{8.5cm}}
    
    \begin{lstlisting} [breaklines=true]
      r=1
      n=5
      lote=3, 2, -5, 1, -1
      watts gastados: 19
    \end{lstlisting} &
    
    \begin{lstlisting} [breaklines=true]
      r=5
      n=3
      lote=1 2 3
      watts gastados: 15
    \end{lstlisting}
    
  \end{tabular}
  
\end{enumerate}

\end{document}

